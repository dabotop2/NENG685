%% LyX 2.2.3 created this file.  For more info, see http://www.lyx.org/.
%% Do not edit unless you really know what you are doing.
\documentclass[english]{article}
\usepackage[latin9]{inputenc}
\usepackage[active]{srcltx}
\usepackage{array}

\makeatletter

%%%%%%%%%%%%%%%%%%%%%%%%%%%%%% LyX specific LaTeX commands.
%% Because html converters don't know tabularnewline
\providecommand{\tabularnewline}{\\}

%%%%%%%%%%%%%%%%%%%%%%%%%%%%%% User specified LaTeX commands.
\usepackage{hyperref}
\usepackage{geometry}
\usepackage{fontenc}
% Set your name here
\def\name{Capt James Bevins}
% The following metadata will show up in the PDF properties
\hypersetup{
  colorlinks = true,
  urlcolor = black,
  pdfauthor = {\name},
  pdftitle = {\name: NENG 685 Syllabus},
  pdfsubject = {NENG 685 Syllabus},
  pdfpagemode = UseNone
}
% Customize geometry
\geometry{
  body={6.5in, 9.0in},
  left=1.0in,
  top=1.0in
}
% Customize page headers
%\pagestyle{myheadings}
%\markright{\name}
\thispagestyle{empty}
% Custom section fonts
\usepackage{sectsty}
\sectionfont{\rmfamily\mdseries\Large}
\subsectionfont{\rmfamily\mdseries\itshape\large}
% Other possible font commands include:
% \rmfamily for roman
% \ttfamily for teletype,
% \sffamily for sans serif,
% \bfseries for bold,
% \scshape for small caps,
% \normalsize, \large, \Large, \LARGE sizes.
% Don't indent paragraphs.
\setlength\parindent{0em}
% Make lists without bullets and compact spacing

\makeatother

\usepackage{babel}
\begin{document}
\begin{center}
{\huge{}NENG 685 - Fall 2017}
\par\end{center}{\huge \par}

\begin{center}
{\huge{}Methods for Neutral Particle Transport}
\par\end{center}{\huge \par}

\bigskip{}

\begin{minipage}[t]{0.5\textwidth}%
Instructor: Capt James Bevins

Office: Bldg 640, Rm 331A

E-mail: jbevins1@afit.edu

Office phone: (937) 255-3636 x4767%
\end{minipage}\hspace{0.15in}%
\begin{minipage}[t]{0.4\textwidth}%
Class: MW, 0800-1000, Bldg 640, Rm 222

Office hours: T/Th, 0900-1000 \& by appt

%Personal Email: \href{mailto:feejames@gmail.com}{feejames@gmail.com}%
\end{minipage}

\section*{Course Description}

This course covers the principal methods used for deterministically solving the Boltzmann transport equation for neutral particles (neutrons and photons). This course presents the fundamental mathematical and computational methods using discretezations in space, energy, and angle.  Iterative methods for efficient solution of the transport problems are explored and analyzed.  Monte Carlo and Discrete Ordinates methods are explicitly developed and applied to shielding and criticality problems of interest.  The course will include both code development and use of existing codes for solving criticality and shielding problems of interest in nuclear engineering.

\section*{Requisites}

NENG 651 - Introduction to Nuclear Physics

\section*{Textbooks}

\textit{FORTRAN 95/2003 For Scientists and Engineers, 3d Ed, }Steven
Chapman, McGraw-Hill, 2008.

\textit{A Primer on Scientific Programming with Python, }Hans Petter Langtangen, 3rd Ed.,
Springer,  2012.

\textit{Effective Computation in Physics, }Anthony Scopatz and Kathryn Huff, O'Reilly,  2015.

\section*{Additional References}

\begin{itemize}
  \item Course notes and handouts: \url{https://github.com/jamesbevins/NENG685}
  \item Free programming lessons: \url{https://software-carpentry.org/lessons/}
  \item Reproducible Research: \url{https://www.practicereproducibleresearch.org/}
  \item Helpful guides: \url{https://github.com/jamesbevins/SurvivingAFIT}
\end{itemize}

\section*{Course Objectives}

At the end of the course, students should:
\begin{itemize}
\item Understand program design, structure, and procedures
\item Develop code to solve problems of interest in nuclear engineering
\item Be able to numerically solve ODEs, PDEs, and systems of equations
\item Be able to perform numerical integration and differentiation
\item Understand the different methods of neutron transport and associated benefits and drawbacks
\item Understand Boltzmann Transport Equation, Diffusion Equation, and the methods used to solve
\item Understand Monte Carlo transport techniques and associated concepts
\item Be familiar with deterministic transport (PENTRAN) and stochastic transport (MCNP) codes
\item Be comfortable operating in the remote high performance computing environments available at AFIT
\end{itemize}

\section*{Grading}

Grading will consist of a variety of metrics: homeworks, "pre-flight" quizzes, and a final project. 
The grading structure reflects that programming and algorithm development can only be achieved through practice. 
The pre-flight quizzes are meant to enhance learning by instilling key foundational ideas prior to use in class.

\begin{itemize}
  \item Homework: 45\%\footnote{Late submissions: -20\% for each day it is late with a maximum of -60\%.}
  \item Pre-flight Quizzes: 25\%\footnote{Pre-flight quizzes are based on the assigned reading/material, and are due by 2359 the day before the class they are due.}
  \item Final Project (code, paper, presentation): 30\% 
\end{itemize}

It is not anticipated that there will be a curve applied.  It is my belief that the grades will sort themselves out if the course assessment material is properly done.  I have no set distribution of grades and have no reservation about everyone earning "As" or any other grade on the spectrum.  Rarely are there enough statistics (i.e. students) in this class to try to impose a normal distribution on the grades and in the AF (and grad school) 90\% of the people are above average. \newline

Extra points - specified in each assignment - may be available for following "best practices" that are above and beyond what is required to complete the assignment.

\section*{Programming Language}

My philosophy is that it is best to know one language really well because different organizations, communities, projects, etc. will require different languages, and it is impossible and often counterproductive to learn them all.  If you know one language well, you can pick up other languages when the need arises. \newline

Consistent with this philosophy, this course will primarily be taught using Python. The goal is to get you proficient in \textit{a language} by the end of the course. However, you are free to use whatever language you feel comfortable with and works for your workflow.  The languages I will be able to help with in terms of decreasing knowledge and/or increasing time since last use: Python, Matlab, FORTRAN, C++, and Java.  There are other resources in the department, online, and in my library for the various languages.

\section*{Academic Integrity}

Academic Integrity - "Uncompromising adherence to a code of ethics, morality, conduct, scholarship, and other values related to academic activity." (AUI 36-2309) \newline

Academic Freedom - "Academic freedom must be tempered by good judgment to refrain from making offensive remarks, unfounded opinions, or irresponsible statements." (AUI 36-2308) \newline

Non-Attribution - "All guest speakers, students, and permanent-party personnel are prohibited from divulging the identity of any particular speaker, whether a guest speaker, faculty member, or student, for the purpose of attributing to that speaker any specific remarks or statements, including but not limited to offensive remarks and irresponsible statements." (AUI 36-2308) \newline

AFIT provides detailed guidance about these policies in the Student Handbook. Lack of knowledge of these policies is not a reasonable explanation for a violation. Questions related to course assignments and the academic honesty policy should be directed to me. \newline

There are also several great publications on your responsibilities as a scientist/engineer and researcher.  I'd recommend at least skimming:

\begin{itemize}
  \item The National Academies Press's \href{https://www.nap.edu/catalog/12192/on-being-a-scientist-a-guide-to-responsible-conduct-in}{"On Being a Scientist"} 
  \item The National Academy of Sciences's \href{https://www.nap.edu/catalog/21896/fostering-integrity-in-research}{"Fostering Integrity in Research"}
\end{itemize}

My policy is that you may work together on homework, but you must \textbf{specifically cite with whom you worked and what you did together}.

\section*{Attendance}

Course credit presumes the student has attended all class sessions in addition to completing all tasks.  Thus, attendance is mandatory in all sessions to graduate.  If students have an extraordinary reason to be absent (example: a sickness, car breakdown, etc.), the student must contact the instructor (in advance if possible) to evaluate if the absence will be excusable prior to continuing the student in the course.

\section*{Course Schedule}

\begin{tabular}{|c|c|c|c|c|c|}
\hline 
\textbf{Date} & \textbf{Assigned} & \textbf{Due} & \textbf{Topic} & \textbf{Reading} & \textbf{Calendar Notes}\tabularnewline
\hline 
Mon 2 Oct &  & PF01 & Intro to Comp & Scopatz Ch 2$^3$ & \tabularnewline
          &  &  & Numerical Error      &  & \tabularnewline
\hline 
Wed 4 Oct & HW01 & PF02 & Interp \& Approx & Scopatz Ch 3-4$^3$ & \tabularnewline
\hline 
Fri 6 Oct & --- & --- & --- & --- & Last Day to Add\tabularnewline
\hline 
Mon 9 Oct & --- & --- & --- & --- &  Columbus Day\tabularnewline
\hline 
Wed 11 Oct &  & HW01/PF03 & Diff \& Int & Scopatz Ch 5$^3$ & \tabularnewline
\hline 
Fri 13 Oct & --- & --- & --- & --- & Drop w/o Record \tabularnewline
\hline 
Mon 16 Oct &  & PF04 & ODEs \& PDEs & Scopatz Ch 6$^3$ & \tabularnewline
\hline 
Wed 18 Oct & HW02 & PF05 & Vectors, Matrices, & Scopatz Ch 9$^3$ &  \tabularnewline
           &  &      & \& Sys of Eqns     &              &  \tabularnewline
\hline 
Mon 23 Oct &  &  & BTE & Course Notes & \tabularnewline
\hline 
Wed 25 Oct & HW03 & HW02 & Diffusion Eqn & Course Notes & \tabularnewline
\hline 
Mon 30 Oct & FP$^4$ &  & DE Discretization & Course Notes & \tabularnewline
\hline 
Wed 1 Nov & HW04 & HW03/PF06 & DE Discretization & Scopaz Ch1 & \tabularnewline
\hline 
Mon 6 Nov &  &  & PENTRAN Part I & PENTRAN Manual & Dr. Sjoden teaches \tabularnewline
\hline 
Wed 8 Nov &  &  & PENTRAN Part II & PENTRAN Manual & Dr. Sjoden teaches \tabularnewline
\hline 
Mon 13 Nov &  & HW04 & Iterative Methods &  & \tabularnewline
\hline 
Wed 15 Nov & HW05 & PF07 & Runge-Kutta & PRKE Notes & \tabularnewline
\hline 
Mon 20 Nov &  &  &  & Monte Carlo & \tabularnewline
\hline 
Wed 22 Nov & HW06 & HW05 & Monte Carlo &  & \tabularnewline
\hline 
Fri 24 Nov & --- & --- & --- & --- & Drop w/o Grade \tabularnewline
\hline 
Mon 27 Nov &  & PF08 & MCNP & TBD & \tabularnewline
\hline 
Wed 29 Nov & HW07 & HW06 & MCNP &  & \tabularnewline
\hline 
Mon 4 Dec &  &  & ADVANTG &  & \tabularnewline
\hline 
Wed 6 Dec &  & HW07 & FP Presentation &  & 15 Min Talk \tabularnewline
\hline 
Tue 12 Dec &  & FP Paper &  &  & \tabularnewline
\hline 
\end{tabular}
$^3$ Or equivalent for different language \newline
$^4$ Final Project

\bigskip{}
\textbf{Disclaimer:}  The course instructor retains the right to adjust the course material and grading process while the course is underway as necessary to maximize the learning experience and assess student learning appropriately.  


\bigskip{}
\begin{center}
\textit{\small{}Last updated: \today}
\par\end{center}{\small \par}

\end{document}
