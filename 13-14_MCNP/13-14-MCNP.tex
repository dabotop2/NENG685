%--------------------------------------------------------------------
% NENG 685 (intro to methods for neutral particle transport)

\documentclass[12pt]{article}
\usepackage[top=1in, bottom=1in, left=1in, right=1in]{geometry}

\usepackage{setspace}
\onehalfspacing

\usepackage{amssymb}
%% The amsthm package provides extended theorem environments
\usepackage{amsthm}
\usepackage{epsfig}
\usepackage{times}
\renewcommand{\ttdefault}{cmtt}
\usepackage{amsmath}
\usepackage{graphicx} % for graphics files
\usepackage{tabu}

% Draw figures yourself
\usepackage{tikz} 

% The float package HAS to load before hyperref
\usepackage{float} % for psuedocode formatting
\usepackage{xspace}

% from Denovo Methods Manual
\usepackage{mathrsfs}
\usepackage[mathcal]{euscript}
\usepackage{color}
\usepackage{array}

\usepackage[pdftex]{hyperref}
\usepackage[parfill]{parskip}

% math syntax
\newcommand{\nth}{n\ensuremath{^{\text{th}}} }
\newcommand{\ve}[1]{\ensuremath{\mathbf{#1}}}
\newcommand{\Macro}{\ensuremath{\Sigma}}
\newcommand{\rvec}{\ensuremath{\vec{r}}}
\newcommand{\vecr}{\ensuremath{\vec{r}}}
\newcommand{\omvec}{\ensuremath{\hat{\Omega}}}
\newcommand{\vOmega}{\ensuremath{\hat{\Omega}}}
\newcommand{\sigs}{\ensuremath{\Sigma_s(\rvec,E'\rightarrow E,\omvec'\rightarrow\omvec)}}
\newcommand{\el}{\ensuremath{\ell}}
\newcommand{\sigso}{\ensuremath{\Sigma_{s,0}}}
\newcommand{\sigsi}{\ensuremath{\Sigma_{s,1}}}
%---------------------------------------------------------------------------
%---------------------------------------------------------------------------
\begin{document}
\begin{center}
{\bf NENG 685, Fall 17 \\
Introduction to MCNP\\
November 17th and 20th, 2017}
\end{center}

These course notes take a different approach to previous notes.
Instead of being explicit, key concepts are listed with references back to the MCNP manual.
This avoids duplication on my part, but more importantly, it gets everyone into the manuals, which is a must for continued use of MCNP.
All of the page numbers listed are the pdf pg \# from the MCNP6 manual (v1.0 Rev0) for easy browsing unless otherwise noted.
This manual is included in the directory for these course notes.

\section*{Learning Objectives}

After the class and assignments related to this material, you should be able to 
\begin{enumerate}
  \item Develop your own geometry for simple radiation transport problems.
  \item Create a source that replicates physical sources using built-ins and basic distributions.
  \item Define tallies for flux and reaction rates.
  \item Modify the simulation physics
  \item Run criticality calculations
  \item Interpret the MCNP output
\end{enumerate}

\section*{Input Structure}

The basic input deck structure is specified as (pg 22):

One line problem title card \\
\textit{blank line delimiter} \\
Cell cards \\
\textit{blank line delimiter} \\
Surface cards \\
\textit{blank line delimiter} \\
Data Cards \\
\textit{blank line delimiter (optional)} \\

The complete MCNP input is referred to as a deck.
Each line in the deck is a card.

The input deck is a bit backwards in terms of how you would actually build a deck from scratch, so I'll meander around the deck in the order I find the most logical.

\section*{Surfaces}

Material definitions are discussed starting on pg. 112.
Some key concepts:

\begin{itemize}
  \item Defining surfaces by equations (pg. 65-67)
  \item Reflective surfaces (pg. 66)
  \item Macrobodies (pg. 73-79) and facets (pg. 79)
\end{itemize}

\section*{Materials}

Surface are discussed starting on pg. 64.
Some key concepts:

\begin{itemize}
  \item ZAID specifiers (pg. 113; ; la-ur-13-21822 pg. 19)
  \item Cross-section libraries (pg. 745; la-ur-13-21822)
  \item Material specification (pg. 113-116)
\end{itemize}

\section*{Cells}

Cells are discussed starting on pg. 62.
Some key concepts:

\begin{itemize}
  \item Cell specification (pg. 62-63)
  \item Surface ``sense"
  \item Splitting/Roulette - i.e. importance (pg. 331-332)
\end{itemize}

\section*{Source Specification}

Sources are discussed starting on pg. 177.
Some key concepts:

\begin{itemize}
  \item General SDEF specification (pg. 177-186)
  \item Information, probability, and bias - SI, SP, SB (pg. 194 - 200)
  \item Built-in distributions (pg. 197 - 199)
\end{itemize}

\section*{Tally Specifications}

Tallies are discussed starting on pg. 233.
Some key concepts:

\begin{itemize}
  \item Standard tally specification (pg. 235-241)
  \item Energy modifiers (pg. 257 - 258)
  \item Angle modifiers (pg. 260 - 21)
  \item Tally multipliers (pg. 262 - 269)
\end{itemize}

\section*{Physics Specifications}

Physics are discussed starting on pg. 126.
Some key concepts:

\begin{itemize}
  \item Mode (pg. 126 - 127)
  \item Particle physics (pg. 127 - 139)
  \item Cut cards (pg. 141 - 144)
  \item Physics models (pg. 147 - 157)
\end{itemize}

\section*{Output}

Output is discussed starting on pg. 405.


\end{document}
