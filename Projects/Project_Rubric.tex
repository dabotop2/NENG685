\NeedsTeXFormat{LaTeX2e}
\documentclass[a4paper, 12 pt]{curve}
%\documentclass[12pt]{article}
%\textwidth=7in
%\textheight=9.5in
%\topmargin=-1in
%\headheight=0in
%\headsep=.5in
%\hoffset=-.85in

%\usepackage[cm]{fullpage}
\usepackage[top=0.75in, bottom=0.75in, left=1in, right=1in]{geometry}
\pagestyle{empty}
\usepackage{tabu}
\parindent 0ex
\usepackage{hyperref}

\renewcommand{\thefootnote}{\fnsymbol{footnote}}
\begin{document}

\begin{center}
{\bf NE 155\\ Analysis Final Project Rubric
}
\end{center}

\setlength{\unitlength}{1in}
\begin{picture}(6,.1) 
\put(0,0) {\line(1,0){6.25}}         
\end{picture}

\renewcommand{\arraystretch}{2}

The final paper should be $\sim$4-6 pages per team member and no longer than 8 pages per team member (journals have page limits too). This may vary based on the specific project; please use your best judgment. Please include these items as clearly labeled sections in the \underline{final report}:
%
\begin{enumerate}
\item \textbf{Introduction:} What are you trying to accomplish and why? Also preview what you are going to talk about.

\item \textbf{Problem Description:} Describe the problem you are solving in this work and explain how it will help you find out the thing you told us about in the introduction.

\item \textbf{Description of Work:} What did you do to perform your analysis? This can include any models you built, data you collected, strategies you needed for evaluation, etc.

\item \textbf{Results:} What did you find out?

\item \textbf{Conclusions:} What do your findings mean? How does that relate to the goal you laid out in the introduction?

\item \textbf{References:} You should have references that you cite in your paper.
\end{enumerate}

\vspace*{1em}
If you want feedback, then you are welcome to turn in a \underline{mid-project report}.  This is not required or graded, but a way for you to get feedback and formal course corrections.  In the \underline{mid-project report} please replace ``Conclusions" with \textbf{Plans for Completion} and keep in mind that ``Results" will be preliminary or possibly empty. The first three sections don't have to be completely polished, but they should at least be very solid drafts. The better they are when I read them the more useful the feedback will be. This should be no longer than 4 pages per team member (this is similar to a conference paper or summary).

\vspace*{2em}
Please include these same items in your \underline{final presentation}. The presentations should be approximately 15 minutes. If a team project is done, the presentation must be clearly separated to highlight each member's work. 

\vspace*{2em}
Finally your \underline{code} will be graded on readability, documentation and testing. In other words, assume someone is going to follow you on this project.  The code and paper should provide all of the details needed without any additional input from you, the author. I will not grade efficiency of the code.

\vspace*{2em}
\textbf{Notes for writing papers properly:}
\begin{itemize}
\item If you include figures, use a Figure number and caption; refer to the figure from within the text according to the IEEE style guide.
\item You may need to number equations and refer to them in the text.
\item \textit{Use section headings for the requested sections.}
\item In the introduction, discuss what is coming up in the paper. 
\item In the conclusions, discuss what you told us in the paper.
\item If you talk about a code (that you didn't write yourself), you need to include a reference for that code. 
\item For the final report, it's a good idea to include enough information for the work to be reproducible. To avoid making the report filled with mundane details you can put some items in an appendix or repository that you reference.  Code documentation also serves this purpose.
\item Common grammar errors: \href{http://www.quickanddirtytips.com/education/grammar/which-versus-that-0}{that vs.\ which}, \href{http://grammarpartyblog.com/2012/01/17/use-versus-utilize/}{use vs.\ utilize}, \href{https://e-gmat.com/blog/gmat-verbal/sentence-correction/idioms/due-to-vs-because-of}{due to vs.\ because of}.
\end{itemize}

\vspace*{2em}
I will use the following rubric for evaluating the paper:

\vspace*{1 em}
\begin{center}
\begin{tabu}{| X | c | c |}\hline
\textbf{Category} & \textbf{Possible Points} & \textbf{Earned Points} \\ \hline \hline
Correct Approach taken & 10 & \\ \hline
Work correctly implements approach & 10 & \\ \hline
Goal, problem solved, and analysis conducted have an appropriate logical flow & 12 & \\ \hline
Conclusions are supported by the results & 10 & \\ \hline
Complete sentences; correct grammar and spelling & 8 & \\ \hline
Sources properly documented & 5 & \\ \hline
Total & 55 & \\\hline
\end{tabu} 
\end{center}

This rubric is for evaluating the presentation:

\vspace*{1 em}
\begin{center}
\begin{tabu}{| X | c | c |}\hline
\textbf{Category} & \textbf{Possible Points} & \textbf{Earned Points} \\ \hline \hline
Topic motivation is clear & 3 & \\ \hline
Explanation of work is understandable, correct, and supports the motivation & 7 & \\ \hline
Results and conclusions are clearly communicated & 6 & \\ \hline
Good presentation skills: eye contact, volume, clarity of slides, etc. & 6 & \\ \hline
Appropriate presentation length & 3 & \\ \hline
Total & 25 & \\\hline
\end{tabu} 
\end{center}

This rubric is for evaluating the code:

\vspace*{1 em}
\begin{center}
\begin{tabu}{| X | c | c |}\hline
\textbf{Category} & \textbf{Possible Points} & \textbf{Earned Points} \\ \hline \hline
Code is clearly documented & 10 & \\ \hline
Test functions exist for each function/class & 10 & \\ \hline
Total & 20 & \\\hline
\end{tabu} 
\end{center}


\end{document}